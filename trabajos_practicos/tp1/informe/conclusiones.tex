Teniendo en cuenta los datos reportados previamente, los resultados expuestos, y considerando además el tiempo requerido para entrenar de cada clasificador, y la dificultad de poder decidir qué parámetros son relevantes con qué valores, concluimos que el mejor clasificador según nuestra experiencia es Random Forest, realizando la reducción de dimensionalidad de la totalidad de features utlizando PCA. 
También creemos que es relevante destacar que arrancamos la etapa de extracción de features intentando buscar patrones y propiedades que separen lo más posible a los SPAMs de los HAMs, creyendo que allí subsistía la magia del asunto y descubrimos a medida que tropezábamos con los resultados, que no siempre esto es lo más eficiente, no solo por la posibilidad de hacer \textit{overfiting}, sino por el sesgo innecesario y contraproducente obtenido al recortar información que puede serle útil al clasificador, al fin y al cabo, ese es su trabajo. Esto no quiere decir que no sea bueno analizar nada, simplemente que llegado un punto el modelo podría no llegar a ser 100\% interpretable.