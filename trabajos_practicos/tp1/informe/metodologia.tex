A partir del dataset de emails etiquetados en SPAM y HAM, realizamos una partición de datos en dos grupos. La primera, compuesta por un 80\% de los emails, fueron separados para usar como set de entrenamiento. Este es el conjunto sobre el cual se llevaron a cabo diferentes análisis para determinar características relevantes y, posteriormente, entrenar los diferentes modelos. El 20\% restante, fue separado como set de testing, permaneciendo sin uso ni análisis, para luego poder contrastar el/los modelos determinados a partir del set de entrenamiento. Ambos grupos fueron conformados de manera aleatoria sobre el 100\% de las muestras utilizando la librería \textit{train\_test\_split} de \textit{scikit-learn}. Para la extracción de features y \textit{grid-search} realizados sobre el dataset de entrenamiento se utilizó la técnica de \textit{cross-validation} para evitar lo máximo posible realizar \textit{overfiting}.